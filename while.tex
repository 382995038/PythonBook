\chapter{While Loops}
A while loop is a function, which needs a boolean statement to run, in order to prit out a list of results. The while loop will print out as many resultts as possible, until the boolean statement stops being true. For example:

\begin{verbatim}
i=0
while (i<5):
    print i
    i=i+1
    
\end{verbatim}

Now, the boolean statement inserted here is "i<5", meaning that "i" must be less than 5. The next function now commands the system to print "i". The result would be:
\begin{verbatim}
0
1
2
3
4
\end{verbatim}

This is because after the computer was demanded to print out "i", the function "i==i+1", was entered, meaning that it should print out all the numbers that make the bolean statement rue, until the number is five, making it false.

There are also many ways you could manipulate this code. A break may be inserted as follows: 
\begin{verbatim}
i=0
while (i<5):
    print i
    i=i+
    if i==4:
        break
    
\end{verbatim}

The result would be:
\begin{verbatim}
0
1
2
3
\end{verbatim}

Excluding "4", because as a result of the break, the computer is now told not to print anything from the number 4.

However, if the break came before "i=i+1" like:
\begin{verbatim}
i=0
while (i<5):
    print i
    if i==4:
        break
    i=i+1
    
\end{verbatim}

The result would be:
\begin{verbatim}
0
1
2
3
4
\end{verbatim}












