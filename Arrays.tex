%% LyX 2.1.4 created this file.  For more info, see http://www.lyx.org/.
%% Do not edit unless you really know what you are doing.
\documentclass[english]{article}
\usepackage[T1]{fontenc}
\usepackage[latin9]{inputenc}
\usepackage{babel}
\usepackage{listings}
\renewcommand{\lstlistingname}{Listing}

\begin{document}

\paragraph{Arrays}

In a nutshell, arrays are lists of data. Most commonly, they are sets
of integers, although they can also be sets of strings. There are
many uses for an array, such as being used as a database to count
occurrences. The simplest arrays are one-dimensional arrays, which
look like this.

\begin{lstlisting}
[0,1,2,3,4]
["a","b","c","d","e"]
\end{lstlisting}


These arrays can be described with several different characteristics.
\begin{enumerate}
\item They are both an array of length 5.
\item They are both one dimensional arrays (trust me, this will make more
sense).
\item The first is a list of integers, and the second is a list of strings.
\end{enumerate}

\subparagraph{Array generation}

Arrays are generated in a program by first creating an empty array,
or by creating an array that is already filled with whatever you'd
like.

\begin{lstlisting}
a = []
savoryfillings = ["meat","cheese","potato"]
\end{lstlisting}



\subparagraph{Adding to an array}

Adding to an array, also known as ``concatenation'', is also a very
simple process, much like generating an array. A simple way to fill
an array with, let's say, zeroes, is with a for loop.

\begin{lstlisting}
a = []
for i in range(5):
	a = a + [0]
print a
#this program will yield [0, 0, 0, 0, 0]
\end{lstlisting}


Note how the zero is in brackets. This will signify to the interpreter
that this zero is intended to be a unit of the array.

Strings can also be concatenated to an array of strings.

\begin{lstlisting}
>>> savoryfillings = ["meat","cheese","potato"] 
>>> savoryfillings = savoryfillings + ["spinach"] 
>>> print savoryfillings 
['meat', 'cheese', 'potato', 'spinach']
\end{lstlisting}



\subparagraph{Referencing certain parts of an array}

Referencing certain parts of an array is rather simple. A simple program
that does such thing and prints the referenced points is displayed
below. Remember that Python starts counting at 0, not 1!

\begin{lstlisting}
>>> a = [1, 2, 3, 4]
>>> x = a[0]
>>> y = a[1]
>>> #referencing points of array is done by (nameofarray)[pointinarray]
>>> print x
>>> print y
1
2
\end{lstlisting}



\subparagraph{The command 'len'}

The 'len' command takes the length of a given array and converts it
into an integer. 

\begin{lstlisting}
>>> a = [1, 2, 3]
>>> x = len(a)
>>> print x
3
\end{lstlisting}


'len' can also be used to do something to every point of an array,
with the use of a for loop.

\begin{lstlisting}
>>> a = [0, 1, 2, 3, 4, 5]
>>> for i in range(len(a)): 	
	a[i] = a[i] + 1

>>> print a 
[1, 2, 3, 4, 5, 6]  
\end{lstlisting}



\subparagraph{Two-/three-dimensional arrays}

Multidimensional arrays can be created through the use of nested for
loops - for example, the program below generates a 5x5 array of zeroes.

\begin{lstlisting}
>>> a = [] 
>>> for i in range(5):     
		b = []     
		for j in range(5):         
			b = b+[0]     
		a = a+[b] 
>>> print a
[[0, 0, 0, 0, 0], [0, 0, 0, 0, 0], [0, 0, 0, 0, 0], [0, 0, 0, 0, 0], [0, 0, 0, 0, 0]]
\end{lstlisting}


Note how this two-dimensional array has five arrays within it, each
with five zeroes.


\subparagraph{Application of arrays}

Arrays can be used to catalog data. This program logs 

\begin{lstlisting}
import random
mastercount = []
for i in range (10):
	mastercount = mastercount + [0]
for i in range (1000):
	x = random.randint(0,9)
	mastercount[x] = mastercount[x]+1
print mastercount
#sample yield of program - [95, 103, 96, 98, 107, 108, 101, 107, 102, 83]
\end{lstlisting}

\end{document}
