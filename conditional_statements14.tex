\chapter{Conditional Statements}



\section{Introduction}
I am going to talk about if and else.
The if statement is used for conditional execution.
It selects exactly one of the suites by evaluating the 
expressions one by one until one is found to be true.
if all of the expressions are false, it will choose else.
And the elif statement. It stands for "else if," which means 
that if the original if statement is false and the elif 
statement is true.An else statement can be combined with an 
if statement. An else statement contains the block of code 
that executes if the conditional expression in the if 
statement resolves to 0 or a false value. 
\begin{verbatim}
a=15
b=10
if a>b:
    print "a>b"
else: print "a<=b"
\end{verbatim}
And the output:
\begin{verbatim}
a>b
\end{verbatim}
if "a" greater than "b",print "a>b".For example,"a' is 15,
"b" is 10,15 is greater than 10,so print"a>b". if "a" is not 
greater than "b",then,choice else,print "a<=b"

\begin{verbatim}
a = 0

while a < 10:
    a = a + 1
    if a > 5:
        print (a,">",5)
    elif a <= 7:
        print (a,"<=",7)
    else:
        print ("Neither test was true")
\end{verbatim}
And the output:
\begin{verbatim}
1 <= 7
2 <= 7
3 <= 7
4 <= 7
5 <= 7
6 > 5
7 > 5
8 > 5
9 > 5
10 > 5 
\end{verbatim}
first,because of a<10.a=1+1 the result should greater than 5
so "a" will from 0 to 9 until a cannot change any more,
so, a=6,7,8,9,10.if not, like a<7 or a=7, print a<=7.so 
1,2,3,4,5 is the answer.
 
  
  
In the Game of Life we updated the grid depending on the number of neighbors of a cell.
\begin{verbatim}
if A[i][j]==0:
    if neigh==3:
        B[i][j]=1
    else:
        B[i][j]=0
else:
    if neigh==2 or neigh==3:
        B[i][j]=1
    else:
        B[i][j]=0
\end{verbatim}
