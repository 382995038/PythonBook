\chapter{Statistics Library}

\section{Introduction}
This statistics library includes eight functions that we can use to deal with a set of data. \\
\textit{Tips}
\begin{itemize}
\item \textit{Additional libraries are needed}
\item \textit{Multiple functions are needed for some operations}
\end{itemize}

\section{Zero}

\begin{equation}
\centering
$$\begin{verbatim}
def zeros(n):
    a=[]
    for i in range(n):
        a=a+[0]
    return a
\end{verbatim}
$$\end{equation}
This function can be used to create a consecutive and repeating array(all elements are the same). \\
\noindent \textit{Example:}
\begin{verbatim}
def zero(n):
    a=[]
    for i in range(n):
        a=a+[0]
    return a

print zero(9)
\end{verbatim} 
\textit{Output:} 
\texttt{ [0, 0, 0, 0, 0, 0, 0, 0, 0] }

\section{Summing an array}
\begin{equation}
$$\begin{verbatim}
def sum_array(a):
    s=0
    for i in range(len(a)):
        s=s+a[i]
    return s
\end{verbatim}
$$\end{equation}
This function can help us to sum up all the elements in an array. \\
\noindent \textit{Example:} 
\begin{verbatim}
a=[1,2,3,4]
def sum_array(a):
    s=0
    for i in range(len(a)):
        s=s+a[i]
    return s
print sum_array(a)
\end{verbatim} 
\textit{Output:}
\texttt{10}

\section{Finding means}
\begin{equation}
$$\begin{verbatim}
def avg(a):
    return sum_array(a)/len(a)
\end{verbatim}
$$\end{equation}
This function can help us to calculate the mean of all the data in an array, and we need assistance of the sum function. \\
\noindent \textit{Example:} 
\begin{verbatim}
a=[23,19,25,21]
def sum_array(a):
    s=0
    for i in range(len(a)):
        s=s+a[i]
    return s

def avg(a):
    return sum_array(a)/len(a)

print avg(a)
\end{verbatim} 
\textit{Output:}
\texttt{22}

\section{Variance}
\begin{equation}
$$\begin{verbatim}
def var(a):
    s=0
    for i in range(len(a)):
        s=s+a[i]**2
    m=avg(a)
    return (s/len(a)-m**2)
\end{verbatim}
$$\end{equation}
This function can help us to find the variance of the data.Also, we need the assistance of the average function. \\
\noindent \textit{Example:}
\begin{verbatim}
a=[23,19,25,21]
def sum_array(a):
    s=0
    for i in range(len(a)):
        s=s+a[i]
    return s

def avg(a):
    return sum_array(a)/len(a)

print avg(a)

def var(a):
    s=0
    for i in range(len(a)):
        s=s+a[i]**2
    m=avg(a)
    return (s/len(a)-m**2)
print var(a)
\end{verbatim}
\textit{Output: 5}

\section{Construct an array of random numbers}
\begin{equation}
$$\begin{verbatim}
def rand_array(n,mini,maxi):
    a=[]
    for i in range(n):
        a=a+[random.uniform(mini,maxi)]
    return a
\end{verbatim}
$$\end{equation}
We can use this function to construct an array filled with random numbers. We need the random library to run the function.Here n represents the number of elements in the array; mini is the minimum value of the elements;maxi is the maximum value

\section{To fill a histogram} \textit{Library "graphics" is needed*}
\begin{equation}
$$\begin{verbatim}
def histogram(mini,maxi,bins,a):
    h=zeros(bins)
    w=(maxi-mini)/bins
    for i in range(len(a)):
        for j in range(bins):
            if (a[i]>(mini+j*w))and a[i]<(mini+(j+1)*w):
                h[j]=h[j]+1
    return h
\end{verbatim}
$$\end{equation}
The four arguments in parenthesis are decisive for the histogram.
Here mini represents the minimum value in the data;
maxi represents the maximum value in the data;
bin represents the number of different groups of data;
a is the number of all the data.

\section{Find the maximum value}
\begin{equation}
$$\begin{verbatim}
def maximum(a):
    m=0
    for i in range(len(a)):
        if a[i]>m:
            m=a[i]
    return m
\end{verbatim}
$$\end{equation}
This funciton can help us to identify the maximum value in the data.

\section{Drawing a histogram} \textit{Library "graphics" is needed}\\
\noindent To draw a histogram, we first need to define a function that can draw a bargragh:
\begin{equation}
$$\begin{verbatim}
def bar(a,win):
    win.setCoords(-1,-1,len(a)+1,maximum(a)+1)
    bl=[]
    tr=[]
    rec=[]
    for i in range(len(a)):
        bl=bl+[Point(i,0)]
        tr=tr+[Point(i+a,a[i])]
        rec=rec+[Rectangle(bl[i],tr[i])]
        rec[i].draw(win)
\end{verbatim}
$$\end{equation} \\
Then we combine the "bar" function and the "histogram" function.\\
\noindent \textit{Here is just an example:}
\begin{equation}
$$\begin{verbatim}
def main():
    win=GraphWin()
    a=zeros(400)
    for i in range(len(a)):
        a[i]=sum_array(rand_array(10,0,1))
    histo=histogram(0.,10.,7,a)
    bar(histo,win)
main()
\end{verbatim}
$$\end{equation}
