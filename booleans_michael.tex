\section{\textbf{Definition:}}

Booleans:

In Python we have the following terms (characters and phrases) for determining if something is "True" or "False." Logic on a computer is all about seeing if some combination of these characters and some variables is True at that point in the program.

and


or


not


!= (not equal)


== (equal)


>= (greater-than-equal)


<= (less-than-equal)


True


False


We use these characters to make the truth or not.


NOT:


not False =	True


not True = False


OR:


True or False = True


True or True = True


False or True = True


False or False = False

AND:


True and False = False


True and True = True


False and True = False


False and False = False


NOT OR:


not (True or False) = False


not (True or True) = False


not (False or True) = False


not (False or False) = True


NOT AND:


not (True and False) = True


not (True and True)	= False


not (False and True) = True


not (False and False) =	True


!=:


1 != 0 = True


1 != 1 = False


0 != 1 = True


0 != 0 = False



==:


1 == 0 = False


1 == 1 = True


0 == 1 = False


0 == 0 = True