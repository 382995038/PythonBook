%% LyX 2.1.4 created this file.  For more info, see http://www.lyx.org/.
%% Do not edit unless you really know what you are doing.
\documentclass[english]{article}
\usepackage[T1]{fontenc}
\usepackage[latin9]{inputenc}
\usepackage{babel}
\usepackage{listings}
\renewcommand{\lstlistingname}{Listing}

\begin{document}

\paragraph{Hash library (hashlib)}

The hash library is a library that comes packaged by default with
Python, and allows for hashes in MD5, SHA-1, and SHA-2 family to be
computed within Python. Hashes are a one-way algorithm that allow
for, say, a document to have its own unique identity. They are commonly
used for this purpose of identification for security of data, and
for proof of work.


\subparagraph{hashlib.sha256(b''string'').hexdigest()}

Obtains the hex digest of a byte string hashed with the SHA-256 algorithm.

\begin{lstlisting}
>>> import hashlib 
>>> hashlib.sha256(b"Final Preparation Document").hexdigest 
'e26e7d89792dbb2812253aebd3208123a9673f3620d2d1cd4f213d604866c738'
\end{lstlisting}



\subparagraph{hashlib.md5(b''string'').hexdigest()}

Obtains the hex digest of a byte string hashed with the MD5 algorithm.
Note that MD5 has trouble with collisions, and is not a very secure
algorithm. SHA-256 is preferable for security.

\begin{lstlisting}
>>> import hashlib 
>>> hashlib.md5(b"Final Preparation Document").hexdigest() 
'584c916050194c4985fef0b0dcca1398'
\end{lstlisting}

\end{document}
