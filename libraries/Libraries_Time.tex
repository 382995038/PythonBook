%% LyX 2.1.4 created this file.  For more info, see http://www.lyx.org/.
%% Do not edit unless you really know what you are doing.
\documentclass[english]{article}
\usepackage[T1]{fontenc}
\usepackage[latin9]{inputenc}
\usepackage{babel}
\usepackage{listings}
\renewcommand{\lstlistingname}{Listing}

\begin{document}

\paragraph{Time library (time)}

The time library is a library that comes default with Python. The
time library uses the computer's clock in order to provide various
commands and functions that revolve around time. In time-based programming,
time starts at a point called the epoch. In the case of most Unix-based
operating systems, the epoch is 00:00 1 January 1970.


\subparagraph{time.time()}

Returns the time in seconds as a floating-point number, since the
epoch.

\begin{lstlisting}
>>> import time
>>> time.time()
1447441291.782
>>> #taken at 14:01:40 13 November 2015
\end{lstlisting}

\end{document}
